\section{EC 4.19}

\textbf{Exercise 19.0.1} (Quasi-linear environment). \textit{Provide the definition of quasi-linear environment and provide an example of social choice function in quasi-linear environment.}\\

\textbf{Exercise 19.0.2} (Properties in quasi-linear environment). \textit{Provide the definition of the following properties in quasi-linear environment and provide an example of social choice function satisfying each property:
\begin{itemize}
\item weak and strict budget balance;
\item allocation efficiency;
\item maximality in the range.\\
\end{itemize}}

\textbf{Exercise 19.0.3} (Dictatorship and quasi-linear environment). \textit{Prove that no social choice function in quasi-linear environment can be dictatorial.}

\section{EC 4.20}

\textbf{Exercise 20.0.1} (Groves mechanisms). \textit{Provide the definition of Groves mechanisms.}\\

\textbf{Exercise 20.0.2} (Groves Theorem). \textit{Provide the statement of Groves theorem and provide a sketch of the proof.}\\

\textbf{Exercise 20.0.3} (Application of Groves mechanisms). \textit{Given a function $h_i (\theta_{-i})$ and a setting (composed of players, their types, valuation function, outcomes), apply the corresponding Groves mechanism. Additional question: prove that the given Groves mechanism is or not individually ration/weakly budget balanced in the given setting.}\\

\textbf{Exercise 20.0.4} (Groves mechanisms and individually rationality). \textit{Is a Groves mechanism always individually rational (in ex post)? Provide, if there exist, two examples: in the first, a Groves mechanism is individually rational (in ex post), and, in the second, a Groves mechanism (potentially different from the previous one) is not individually rational (in ex post)?}\\

\textbf{Exercise 20.0.5} (Groves mechanisms and weak budget balance). \textit{Is a Groves mechanism always weak budget balance (in ex post)? Provide, if there exist, two examples: in the first, a Groves mechanism is weak budget balance (in ex post), and, in the second, a Groves mechanism (potentially different from the previous one) is not weak budget balance (in ex post)?}\\

\textbf{Exercise 20.0.6} (Weighted Groves mechanisms). \textit{Provide the definition of weighted Groves mechanisms.}\\

\textbf{Exercise 20.0.7} (Application of weighted Groves mechanisms). \textit{Given a function $h_i (\theta_{-i})$ and a setting (com-posed of players, their types, valuation function, outcomes, and weights), apply the corresponding weighted Groves mechanism. Additional question: prove that the given weighted Groves mechanism is or not individually ration/weakly budget balanced in the given setting.}\\

\textbf{Exercise 20.0.8} (VCG mechanism). \textit{Provide the definition of VCG mechanism.}\\

\textbf{Exercise 20.0.9} (Application of VCG mechanisms). \textit{Given a setting (composed of players, their types, valuation function, outcomes), apply the VCG mechanism. Additional question: prove that the VCG is or not individually ration/weakly budget balanced in the given setting.}\\

\textbf{Exercise 20.0.10} (VCG mechanism and individually rationality). \textit{Is the VCG mechanism always individually rational (in ex post)? Provide, if there exist, two examples: in the first, the VCG mechanism is individually rational (in ex post), and, in the second, the VCG mechanism (potentially different from the previous one) is not individually rational (in ex post)?}\\

\textbf{Exercise 20.0.11} (VCG mechanism and weak budget balance). \textit{Is VCG mechanism always weak budget balance(in ex post)? Provide, if there exist, two examples: in the first, the VCG mechanism is weak budget balance(in ex post), and, in the second, the VCG mechanism (potentially different from the previous one) is not weak budget balance (in ex post)?}\\

\section{EC 4.21}

\textbf{Exercise 21.0.1} (Redistribution function). \textit{Provide the definition of redistribution function.}\\

\textbf{Exercise 21.0.2} (Cavallo’ redistribution). \textit{Provide the definition of Cavallo’s redistribution function.}\\

\textbf{Exercise 21.0.3} (VCG with Cavallo’ redistribution properties). \textit{Describe and prove the properties of the VCG mechanism with the Cavallo’s redistribution function.}\\

\textbf{Exercise 21.0.4} (Application of Cavallo’ redistribution). \textit{Given a setting and a mechanism, apply the Cavallo’s redistribution function.}\\

\textbf{Exercise 21.0.5} (Strict-budget balanced mechanisms). \textit{Provide a mechanism that is individually rational,weakly budget balanced, DSIC, and strict budget balanced. Can Groves mechanisms be strict budget balanced?}\\

\section{EC 4.22}

\textbf{Exercise 22.0.1} (Single-parameter linear environment). \textit{Provide the definition of single-parameter linear environment.}\\

\textbf{Exercise 22.0.2} (Weak monotonicity). \textit{Provide the definition of weak monotonicity for single-parameter linear environment.}\\

\textbf{Exercise 22.0.3} (Myerson mechanisms). \textit{Provide the definition of Myerson mechnaisms for single-parameter linear environment.}\\

\textbf{Exercise 22.0.4} (Myerson theorem). \textit{Provide the statement of the Myerson theorem for dominant-strategy incentive compatibility in single-parameter linear environment and prove it.}\\

\section{EC 4.23}

\textbf{Exercise 23.0.1} (Knapsack auction). \textit{Provide the definition of knapsack auction.}\\

\textbf{Exercise 23.0.2} (Knapsack auction approximation algorithm). \textit{Describe a monotone algorithm to approximate the optimal allocation of the knapsack auction with a ratio of $\frac{1}{2}$ and prove the theoretical bound on the approximation ratio.}\\

\textbf{Exercise 23.0.3} (Application of Knapsack auction). \textit{Given a knapsack auction setting, find the optimal allocation returned by the VCG mechanism and find the allocation returned by the $\frac{1}{2}$ approximation monotone algorithm.}\\

\section{EC 4.24}

\textbf{Exercise 24.0.1} (Combinatorial auction). \textit{Provide the definition of combinatorial auction.}\\

\textbf{Exercise 24.0.2} (Combinatorial auction approximation algorithm). \textit{Describe a monotone algorithm to approximate the optimal allocation of a combinatorial auction.}\\

\textbf{Exercise 24.0.3} (Application of combinatorial auction). \textit{Given a setting of combinatorial auction, find the allocation returned by the $\frac{1}{\sqrt{|I|}}$-approximation monotone algorithm, where is the set of items. Additional question: find the Myerson payments when the approximation algorithm is used.}

\section{EC 4.25}

\textbf{Exercise 25.0.1} (Double auction). \textit{Provide the formal model of double auction.}\\

\textbf{Exercise 25.0.2} (McAfee mechanism). \textit{Describe the McAfee mechanism for double auctions and show that it is a particular Myerson mechanism.}\\

\textbf{Exercise 25.0.3} (Application of double auction). \textit{Given a setting of double auction, apply the McAfee mechanism, returning the allocation chosen by the mechanism and the the payments to the players.}

\section{EC 4.26}

\textbf{Exercise 26.0.1} (Sponsored search auction). \textit{Provide the formal model of sponsored search auction.}\\

\textbf{Exercise 26.0.2} (VCG mechanism). \textit{Describe the VCG mechanism.}\\

\textbf{Exercise 26.0.3} (Application of the VCG mechanism). \textit{Given a setting of sponsored search auctions, apply the VCG mechanism, returning the allocation chosen by the mechanism and the pay-per-click payments to the players.}\\

\textbf{Exercise 26.0.4} (Application of the VCG mechanism with estimated qualities). \textit{Given a setting of sponsored search auctions with estimated qualities, apply the VCG mechanism, returning the allocation chosen by the mechanism and the pay-per-click payments to the players.}\\

\section{EC 4.27}

\textbf{Exercise 27.0.1} (Dominant strategy implementation). \textit{Given:
\begin{itemize}
\item a set of outcomes,
\item a set of players,
\item a finite set of types per player,
\item the valuation function of every player,
\end{itemize}
write in AMPL the MOD file and the DAT file to design in automatic fashion a mechanism that may be:
\begin{itemize}
\item truthful in dominant strategies,
\item individually rational,
\item weakly budget balanced,
\item allocatively efficient,
\end{itemize}
and that might maximize/minimize the revenue of the auctioneer.}\\

\textbf{Exercise 27.0.2} (Bayes-Nash implementation). \textit{Given:
\begin{itemize}
\item a set of outcomes,
\item a set of players,
\item a finite set of types per player,
\item the valuation function of every player,
\end{itemize}
write in AMPL the MOD file and the DAT file to design in automatic fashion a mechanism that may be:
\begin{itemize}
\item truthful in Bayes-Nash equilibrium,
\item individually rational,
\item weakly budget balanced,
\item allocatively efficient,
\end{itemize}
and that might maximize/minimize the revenue of the auctioneer.}\\